% !TeX program = xelatex
\documentclass{beamer}
\usepackage{etex} % fixes new-dimension error
%-------------- template --------------------------------------------------
\usetheme{metropolis}
\metroset{block=fill}
%\usetheme{Boadilla}

% Configuring the foot line
\setbeamertemplate{footline}
{
  \leavevmode%
  \hbox{%
  \begin{beamercolorbox}[wd=.4\paperwidth,ht=2.25ex,dp=1ex,center]{author in head/foot}%
    \usebeamerfont{author in head/foot}\insertshortauthor
  \end{beamercolorbox}%
  \begin{beamercolorbox}[wd=.5\paperwidth,ht=2.25ex,dp=1ex,center]{title in head/foot}%
    \usebeamerfont{title in head/foot}\insertsection
  \end{beamercolorbox}%
  \begin{beamercolorbox}[wd=.1\paperwidth,ht=2.25ex,dp=1ex,right]{date in head/foot}%
    \insertframenumber{} / \inserttotalframenumber\hspace*{2ex} 
  \end{beamercolorbox}}%
  \vskip0pt%
}
% No configuration symbols
\makeatother
\setbeamertemplate{navigation symbols}{}
%----------------------------------------------------------------------------
\usepackage{graphicx,amsmath}
\usepackage{stmaryrd} % cf. interleave
\usepackage{booktabs}
\usepackage{amscd}
\usepackage{multicol}
\usepackage[absolute,overlay]{textpos}
\usepackage{alltt}
\usepackage{proof}
\usepackage{subcaption}
\usepackage{tikz}
\usepackage{tikz-cd}
\usepackage[new]{old-arrows}
\usepackage[all]{xy}
\usepackage{pgfplots}
\usepackage{textcomp}
\usepackage{listings}
\usetikzlibrary{arrows.meta, calc, fit, tikzmark}

\AtBeginSection[]
{
    \begin{frame}
        \frametitle{Table of Contents}
        \tableofcontents[currentsection]
    \end{frame}
}
\author[Renato Neves]{Renato Neves}

% logos of institutions
\titlegraphic{
  \begin{textblock*}{5cm}(7.8cm,7.45cm)
     \includegraphics[scale=0.044]{../images/uminho.png}\hspace*{.85cm}~%
  \end{textblock*}
  \begin{textblock*}{5cm}(9.8cm,7.45cm)
    \includegraphics[scale=0.4]{../images/haslab.pdf}
  \end{textblock*}
}

% code
\lstset{
showstringspaces=false,
keywordstyle=\color{blue},
basicstyle=\fontsize{08.5}{10}\ttfamily,
emph={while,do,if,diff,for,exit,blue,unif,then,else,wait,bernoulli,exp,normal,sqrt,cos,';'},emphstyle=\color{blue},
breaklines=true,
escapeinside={*@}{*@}
}

\input{macros}


% No date
\date{}


\begin{document}

\title{Lecture 3}

\frame[plain]{\titlepage}

\section{Overview}

\begin{frame}{Recall \dots}

        Program equivalence often too strict

        \bigskip
        \pause
        Goal: lift this concept to the \alert{metric} realm

        \bigskip
        \pause
        Brings \alert{functional analysis} to programming languages
        \begin{itemize}
                \item Cauchy sequences (of terms)
                \item Lipschitz continuity (of term constructs)
                \item \dots
        \end{itemize}

\end{frame}

\begin{frame}{The genesis: metric equality}

        Equations labelled by elements of $[0,\infty]$

        \bigskip
        $t =_\epsilon s$ reads as ``$t$ and $s$ are \alert{at most}
        at distance $\epsilon$"

\end{frame}

\begin{frame}{The genesis: metric equality}

        \begin{example}[real-time computation]
                $t =_\epsilon s \, $ : difference of execution times 
                does not exceed $\epsilon$
        \end{example}

        \bigskip
        \pause
        \begin{example}[probabilistic computation]
                $t =_\epsilon s \,$ : diff. of prob.
                w.r.t. any observation does not exceed $\epsilon$
        \end{example}

        \bigskip
        \pause
        \begin{example}[process algebra]
                $t =_{2^{-n}} s \,$ : same behaviour
                in the first $n$ computational steps
        \end{example}
\end{frame}

\begin{frame}{Today}

        The metric equational system

        \medskip
        Examples of metric reasoning 

        \medskip
        The case for linearity

        \medskip
        The challenge of a categorical semantics
\end{frame}

\section{Metric equational system}

\begin{frame}{The equational system: a first glimpse}

        $t =_\epsilon s$ reads as ``$t$ and $s$ are \emph{at most}
        at distance $\epsilon$"

        \bigskip
        Metric generalisations of eq. laws emerge; others 
        become apparent  

        \bigskip
        \begin{examples}
                \[
                        \infer[(trans)]{t =_{q + r} u}
                        {t =_q s \qquad s =_r u}
                        \hspace{2cm}
                        \infer[(weak)]{t =_r s}
                        {t =_q s \quad q \leq r} 
                \]
                \[
                        \infer[(refl)]{t =_0 t}{}
                \]
        \end{examples}

\end{frame}

\begin{frame}{For a truly complete system \dots}
        We need to
        \begin{itemize}
                \item integrate metric equational laws
                        \\[5pt]
                \item integrate system of Lecture 1 
                        \\[5pt]
                \item dictate how metrics interact with term constructs
        \end{itemize}
\end{frame}

\section{Categorical semantics}

\end{document}
