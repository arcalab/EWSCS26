% !TeX program = xelatex
\documentclass{beamer}
\usepackage{etex} % fixes new-dimension error
%-------------- template --------------------------------------------------
\usetheme{metropolis}
\metroset{block=fill}
%\usetheme{Boadilla}

% Configuring the foot line
\setbeamertemplate{footline}
{
  \leavevmode%
  \hbox{%
  \begin{beamercolorbox}[wd=.4\paperwidth,ht=2.25ex,dp=1ex,center]{author in head/foot}%
    \usebeamerfont{author in head/foot}\insertshortauthor
  \end{beamercolorbox}%
  \begin{beamercolorbox}[wd=.5\paperwidth,ht=2.25ex,dp=1ex,center]{title in head/foot}%
    \usebeamerfont{title in head/foot}\insertsection
  \end{beamercolorbox}%
  \begin{beamercolorbox}[wd=.1\paperwidth,ht=2.25ex,dp=1ex,right]{date in head/foot}%
    \insertframenumber{} / \inserttotalframenumber\hspace*{2ex} 
  \end{beamercolorbox}}%
  \vskip0pt%
}
% No configuration symbols
\makeatother
\setbeamertemplate{navigation symbols}{}
%----------------------------------------------------------------------------
\usepackage{graphicx,amsmath}
\usepackage{stmaryrd} % cf. interleave
\usepackage{booktabs}
\usepackage{amscd}
\usepackage{multicol}
\usepackage[absolute,overlay]{textpos}
\usepackage{alltt}
\usepackage{proof}
\usepackage{subcaption}
\usepackage{tikz}
\usepackage{tikz-cd}
\usepackage[new]{old-arrows}
\usepackage[all]{xy}
\usepackage{pgfplots}
\usepackage{textcomp}
\usepackage{listings}
\usetikzlibrary{arrows.meta, calc, fit, tikzmark}

\AtBeginSection[]
{
    \begin{frame}
        \frametitle{Table of Contents}
        \tableofcontents[currentsection]
    \end{frame}
}
\author[Renato Neves]{Renato Neves}

% logos of institutions
\titlegraphic{
  \begin{textblock*}{5cm}(7.8cm,7.45cm)
     \includegraphics[scale=0.044]{../images/uminho.png}\hspace*{.85cm}~%
  \end{textblock*}
  \begin{textblock*}{5cm}(9.8cm,7.45cm)
    \includegraphics[scale=0.4]{../images/haslab.pdf}
  \end{textblock*}
}

% code
\lstset{
showstringspaces=false,
keywordstyle=\color{blue},
basicstyle=\fontsize{08.5}{10}\ttfamily,
emph={while,do,if,diff,for,exit,blue,unif,then,else,wait,bernoulli,exp,normal,sqrt,cos,';'},emphstyle=\color{blue},
breaklines=true,
escapeinside={*@}{*@}
}

\input{macros}


% No date
\date{}


\begin{document}

\title{Lecture 2}

\frame[plain]{\titlepage}

\section{Overview}

\begin{frame}{Today}
        The bare essentials of category theory


        \medskip
        Useful for providing a semantics to our calculus
\end{frame}

\begin{frame}{Why semantics?}

        Links the calculus to well-known mathematical models
                
        \medskip
        Also useful for obtaining new insights about the calculus
\end{frame}

\section{Category Theory}

\begin{frame}{Category theory in a nutshell}

        The mathematics of mathematics

        \medskip
        A way of proving results once and for all

        \medskip
        \dots
\end{frame}

\begin{frame}{A category}

        Consists of a collection of things (aka \alert{objects}) $A,B,C$ \dots

        \medskip
        and arrows (aka \alert{morphisms}) $f : A \to B$ connecting these things

        \pause
        \bigskip
        There is an \alert{identity} arrow $\id : A \to A$ for each $A$

        \pause
        \bigskip
        Arrows are \alert{composable} ie
        \[
                \infer{
                        g \comp f : A \to C
                }{
                        f : A \to B \qquad g : B \to C}
        \]
        \pause
        (composition is associative and has $\id$ as neutral element)
\end{frame}

\begin{frame}{Examples}
        \begin{itemize}
                \item Sets and functions ($\Set$) \\[10pt]
                \item Vector spaces and linear functions ($\CVec$) \\[10pt]
                \item Metric spaces and non-expansive functions ($\Met$) \\[10pt]
                \item Reals numbers and the $\leq$ relation ($\CR$)  \\[10pt]
                \item $\Nats$-indexed metric spaces and $\Nats$-indexed maps
                        $([\Nats,\Met])$
        \end{itemize}

        \pause
        \vfill
        \begin{block}{Exercise}
                Show that all these are indeed examples of a category 
        \end{block}
\end{frame}

\begin{frame}{A symmetric monoidal category}

        A monoidal category allows
        \begin{itemize}
                \item \alert{pairing} two objs. $A$ and $B$ into a single one $A
                        \otimes B$ 
                \item and analogously for morphisms, ie
                        \[
                                \infer{
                                        f \otimes g : A \otimes B \to A' \otimes B'
                                }{
                                        f : A \to A' \qquad g : B \to B'
                                }
                        \]
                \item It has an obj. $I$ such that $i : I \otimes A \stackrel{\cong}{\to} A$ for
                        every obj. $A$
        \end{itemize}

        {\scriptsize (of course a number of coherence laws also need to hold)}

        \bigskip
        There is also a \alert{symmetry} morphism $\mathrm{sw} : A \otimes B
        \stackrel{\cong}{\to} B \otimes A$
\end{frame}

\begin{frame}{Examples}

        All previous examples of a category are also symm. monoidal

        \bigskip
        \begin{block}{Exercise}
                Show that such is indeed the case 
        \end{block}
\end{frame}



\end{document}
